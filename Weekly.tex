\documentclass[12pt]{article}

% --- PACKAGES ---

% 'geometry' is used to set the page margins.
% 'margin=1in' gives us 1-inch margins on all sides,
% which looks clean and uses the page well.
\usepackage[margin=1in]{geometry}

% 'enumitem' gives us advanced control over lists (like changing font size)
\usepackage{enumitem}


% --- PAGE STYLE ---
% '\pagestyle{empty}' removes the page number from the bottom of the page.
\pagestyle{empty}


% --- DOCUMENT START ---
% The 'document' environment is where all your visible content goes.
% You should only have ONE \begin{document} and ONE \end{document} in your file.
\begin{document}

% --- 1. THE HEADING ---
% 'begin{center}' centers everything inside it.
% '\Huge' makes the text very large.
% '\textbf' makes the text bold.
\begin{center}
    \Huge \textbf{Weekly Tasks}
\end{center}

% --- 2. THE LINE ---
% '\rule' creates a horizontal line.
% '\linewidth' makes the line span the full width of the text.
% '1.5pt' sets the thickness of the line.
\rule{\linewidth}{1.5pt}

% --- 3. THE DATE ---
% '\vspace{5pt}' adds a small amount of vertical space (padding).
% '\today' is a special command that automatically inserts the current date.
\vspace{5pt}
\begin{center}
    \large \today
\end{center}
\vspace{20pt} % Adds a larger space before the list starts


% --- 4. THE TASK LIST ---
%
% 'begin{itemize}' starts a bulleted list.
% We removed the font settings from the options here.
%
\begin{itemize}

    % NEW METHOD: Set the font size *inside* the list.
    % This command applies to EVERYTHING that follows in this 'itemize'
    % environment, including all dots and all text.
    % I've set it to 50pt, which is very large.
    \fontsize{20}{20}\selectfont

    % Each task must start with the '\item' command.
    \item Do two LeetCode/CodeForces Questions and learn a concept\vfill 

    \item 2 Lectures of Unity Course / practice\vfill 

    \item 2 lectures of Source Filmmaker\vfill 

    \item Optional learn CyberSecurity (hacking)\vfill 

    \item learn Web Dev (if it is viable as a side gig) \vfill 


\end{itemize}


% --- DOCUMENT END ---
% This must be the very last command in your .tex file.
\end{document}

